\documentclass[headings=standardclasses, headings=big]{scrreprt}

\input{preamble}

\usepackage[portuguese]{babel}
\usepackage[a4paper,tmargin=1cm]{geometry}

\usepackage{graphics}
\usepackage{listings}

\title{Sucuri}
\subtitle{Uma linguagem baseada em Python}
\author{João Paulo T. I. Z., Ranieri S. A., William K. A.}
\date{\today}

\begin{document}

\maketitle

\clearpage

\section{A linguagem}

A linguagem é planejada tendo como base algumas ideias de Python, Javascript e
Haskell. Para geração do analisador léxico, foi utilizada as ferramentas FLEX
(para especificação do léxico) e BISON (para gerar o código-fonte do
analisador).

Exemplo de código válido na linguagem Sucuri:

\lstinputlisting{../../examples/geometry.scr}

\clearpage

\section{Especificação Sintática}

\lstinputlisting{../../sucuri.ebnf}

\clearpage

\section{Arquivos}

Os arquivos FLEX e BISON são respectivamente \texttt{sucuri.l} e
\texttt{sucuri.y}. Exemplos de programas válidos se encontram na pasta
\texttt{examples/}. O código fonte do analisador é \texttt{sucuri.yy.c}.  Os
logs de saída aplicados no exemplo \texttt{examples/geometry.scr} estão no
arquivo \texttt{geometry-parse.ylog}.

\end{document}
