\documentclass[headings=standardclasses, headings=big]{scrreprt}

\input{preamble}

\usepackage[portuguese]{babel}
\usepackage[a4paper,tmargin=1cm]{geometry}

\usepackage{graphics}
\usepackage{listings}

\title{Sucuri}
\subtitle{Uma linguagem baseada em Python}
\author{João Paulo T. I. Z., Ranieri S. A., William K. A.}
\date{\today}

\begin{document}

\maketitle

\clearpage

\section{A linguagem}

A linguagem é planejada tendo como base algumas ideias de Python, Javascript e
Haskell. Para geração do analisador léxico, foi utilizada as ferramentas FLEX
(para especificação do léxico) e BISON (para gerar o código-fonte do
analisador).

Exemplo de código válido na linguagem Sucuri:

\lstinputlisting{../examples/geometry.scr}

\clearpage

\section{Especificação Léxica}

Inicialmente são definidas algumas regex de apoio:

\begin{verbatim}
D [0-9]                         % Reconhece dígitos

L [a-zA-Z_!@$?]                 % Reconhece qualquer símbolo
                                % possível em um identificador

NO_SQUOTE_STRING_LITERAL [^']*  % Qualquer _string literal_
                                % que não possua aspas simples

NO_DQUOTE_STRING_LITERAL [^"]*  % Qualquer _string literal_
                                % que não possua aspas duplas
\end{verbatim}

Além de duas funções, \texttt{count()}, que realiza contagem de colunas para
gerar uma melhor mensagem de erro (caso ocorra), e \texttt{indent\_level()},
que informa o nível de identação atual.

\subsection{Identificadores}

Identificadores são compostos por qualquer sequência de \texttt{L} ou
\texttt{D} não separados por espaços, podendo conter ``.'' (não no início, no
final nem sucedidos por dígitos).

\subsection{Literais}

\begin{minipage}{\textwidth}
São assumidos como literais de inteiros qualquer construção de somente dígitos:

\begin{verbatim}

# Inteiros válidos:

1
10
0
0000  % Tratado como 0
300
-10   % É reconhecido o "10" como literal inteiro e o "-" como operador unário
      % operado sobre o "10"

\end{verbatim}

Assim, sua \textit{regex} se torna \texttt{\{D\}+} (1 ou mais dígitos
consecutivos).
\end{minipage}

\begin{minipage}{\textwidth}
São assumidos como ponto-flutuante todo literal composto por números e que
tenha um ``.'' no início, meio ou fim do literal:

\begin{verbatim}

1   % Inteiro
1.  % Float
.1  % Float
-1. % "1." reconhecido como literal float, unário "-" operado em "1."

.   % Erro léxico

\end{verbatim}

Assim, sua \textit{regex} é separada em duas:

\begin{itemize}
    \item \texttt{"."\{D\}+} --- Reconhece \textit{floats} iniciados em ".";
    \item \texttt{\{D\}+"."\{D\}*} --- Reconhece \textit{floats} com "." no
        meio ou final;
\end{itemize}
\end{minipage}

\vspace{1em}

\begin{minipage}{\textwidth}
    \textit{String literals} são compostos de qualquer sequência de caracteres que:
    \begin{itemize}
        \item Estão entre aspas simples (') e não possuem outra aspa simples no
            meio (reconhecido pela \textit{regex}
            \texttt{"'"\{NO\_SQUOTE\_STRING\_LITERAL\}"'"}).
        \item Estão entre aspas dupla (") e não possuem outra aspa dupla no
            meio (reconhecido pela \textit{regex}
            \texttt{"\textbackslash\char`\""\{NO\_DQUOTE\_STRING\_LITERAL\}"\textbackslash\char`\""}).
    \end{itemize}
\end{minipage}

Operadores:

\begin{verbatim}
% Unários
not
-

% Comparativos
!=
=
<
<=
>
>=


% Matemáticos
+
-
*
**
/

% Lógicos
and
or
xor

% Outros
(
)

\end{verbatim}

Palavras reservadas:

\begin{verbatim}

% Estruturas de controle
class  % Define um novo tipo
if
else
for
while

% Retornos
return
throw

as     % Serve para alias
catch  % Captura exceções por throw
export % Define o elemento como público
from   % Para importação parcial de um módulo
import % Para importar um módulo
in     % Para iteração (for i in set)
let    % Definição
\end{verbatim}

Há também a definição de elipse (\texttt{...}) para parâmetros variádicos.

\clearpage

\section{Grafo de sintaxe e especificação EBNF}

\begin{minipage}{\textwidth}
  \protect\hypertarget{code}{}{code:}

  \includegraphics[width=3.58333in,height=1.18750in]{diagram/code.png}

  \begin{verbatim}
  code     ::= imports* ( command | definition )+
  \end{verbatim}

  Sem referências

\end{minipage}

\begin{minipage}{\textwidth}
  \protect\hypertarget{literal}{}{literal:}

  \includegraphics[width=2.33333in,height=1.29167in]{diagram/literal.png}

  \begin{verbatim}
  literal  ::= INTEGER_LITERAL
           | FLOAT_LITERAL
           | STRING_LITERAL
  \end{verbatim}

  Referenciado por:

  \begin{itemize}
      \tightlist%%
    \item
      \protect\hyperlink{function_call}{function\_call}
    \item
      \protect\hyperlink{operator}{operator}
  \end{itemize}

\end{minipage}

\begin{minipage}{\textwidth}
  \protect\hypertarget{condition}{}{condition:}

  \includegraphics[width=1.35417in,height=0.37500in]{diagram/condition.png}

  \begin{verbatim}
  condition
         ::= operator
  \end{verbatim}

  Referenciado por:

  \begin{itemize}
      \tightlist%
    \item
      \protect\hyperlink{if_statement}{if\_statement}
    \item
      \protect\hyperlink{while_statement}{while\_statement}
  \end{itemize}

  \clearpage
\end{minipage}

\begin{minipage}{\textwidth}
  \protect\hypertarget{operator}{}{operator:}

  \includegraphics[width=3.43750in,height=2.66667in]{diagram/operator.png}

  \begin{verbatim}
  operator ::= IDENTIFIER
           | literal
           | '(' equality_expr ( ( AND | OR | XOR ) equality_expr )* ')'
  \end{verbatim}

  Referenciado por:

  \begin{itemize}
      \tightlist%
    \item
      \protect\hyperlink{assignment_expr}{assignment\_expr}
    \item
      \protect\hyperlink{attr_decl}{attr\_decl}
    \item
      \protect\hyperlink{command}{command}
    \item
      \protect\hyperlink{condition}{condition}
    \item
      \protect\hyperlink{for_statement}{for\_statement}
    \item
      \protect\hyperlink{unary_expr}{unary\_expr}
  \end{itemize}

\end{minipage}

\begin{minipage}{\textwidth}
  \protect\hypertarget{unary_operator}{}{unary\_operator:}

  \includegraphics[width=1.47917in,height=0.83333in]{diagram/unary_operator.png}

  \begin{verbatim}
  unary_operator
         ::= NOT
           | '-'
  \end{verbatim}

  Referenciado por:

  \begin{itemize}
      \tightlist%
    \item
      \protect\hyperlink{unary_expr}{unary\_expr}
  \end{itemize}

\end{minipage}

\begin{minipage}{\textwidth}
  \protect\hypertarget{unary_expr}{}{unary\_expr:}

  \includegraphics[width=3.16667in,height=0.70833in]{diagram/unary_expr.png}

  \begin{verbatim}
  unary_expr
         ::= unary_operator? operator
  \end{verbatim}

  Referenciado por:

  \begin{itemize}
      \tightlist%
    \item
      \protect\hyperlink{exponential_expr}{exponential\_expr}
  \end{itemize}

\end{minipage}

\begin{minipage}{\textwidth}
  \protect\hypertarget{exponential_expr}{}{exponential\_expr:}

  \includegraphics[width=1.93750in,height=0.83333in]{diagram/exponential_expr.png}

  \begin{verbatim}
  exponential_expr
         ::= unary_expr ( POW unary_expr )*
  \end{verbatim}

  Referenciado por:

  \begin{itemize}
      \tightlist%
    \item
      \protect\hyperlink{multiplicative_expr}{multiplicative\_expr}
  \end{itemize}

\end{minipage}

\begin{minipage}{\textwidth}
  \protect\hypertarget{multiplicative_expr}{}{multiplicative\_expr:}

  \includegraphics[width=2.29167in,height=1.29167in]{diagram/multiplicative_expr.png}

  \begin{verbatim}
  multiplicative_expr
         ::= exponential_expr ( ( '*' | '/' ) exponential_expr )*
  \end{verbatim}

  Referenciado por:

  \begin{itemize}
      \tightlist%
    \item
      \protect\hyperlink{additive_expr}{additive\_expr}
  \end{itemize}

\end{minipage}

\begin{minipage}{\textwidth}
  \protect\hypertarget{additive_expr}{}{additive\_expr:}

  \includegraphics[width=2.37500in,height=1.29167in]{diagram/additive_expr.png}

  \begin{verbatim}
  additive_expr
         ::= multiplicative_expr ( ( '+' | '-' ) multiplicative_expr )*
  \end{verbatim}

  Referenciado por:

  \begin{itemize}
      \tightlist%
    \item
      \protect\hyperlink{relational_expr}{relational\_expr}
  \end{itemize}

\end{minipage}

\begin{minipage}{\textwidth}
  \protect\hypertarget{REL_OP}{}{REL\_OP:}

  \includegraphics[width=1.39583in,height=1.75000in]{diagram/REL_OP.png}

  \begin{verbatim}
  REL_OP   ::= '>'
           | '<'
           | LE
           | GE
  \end{verbatim}

  Referenciado por:

  \begin{itemize}
      \tightlist%
    \item
      \protect\hyperlink{relational_expr}{relational\_expr}
  \end{itemize}

\end{minipage}

\begin{minipage}{\textwidth}
  \protect\hypertarget{relational_expr}{}{relational\_expr:}

  \includegraphics[width=2.06250in,height=0.83333in]{diagram/relational_expr.png}

  \begin{verbatim}
  relational_expr
         ::= additive_expr ( REL_OP additive_expr )*
  \end{verbatim}

  Referenciado por:

  \begin{itemize}
      \tightlist%
    \item
      \protect\hyperlink{equality_expr}{equality\_expr}
  \end{itemize}

\end{minipage}

\begin{minipage}{\textwidth}
  \protect\hypertarget{equality_expr}{}{equality\_expr:}

  \includegraphics[width=2.14583in,height=1.29167in]{diagram/equality_expr.png}

  \begin{verbatim}
  equality_expr
         ::= relational_expr ( ( '=' | NE ) relational_expr )*
  \end{verbatim}

  Referenciado por:

  \begin{itemize}
      \tightlist%
    \item
      \protect\hyperlink{operator}{operator}
  \end{itemize}

\end{minipage}

\begin{minipage}{\textwidth}
  \protect\hypertarget{imports}{}{imports:}

  \includegraphics[width=2.16667in,height=0.83333in]{diagram/imports.png}

  \begin{verbatim}
  imports  ::= module_import
           | feature_import
  \end{verbatim}

  Referenciado por:

  \begin{itemize}
      \tightlist%
    \item
      \protect\hyperlink{code}{code}
  \end{itemize}

\end{minipage}

\begin{minipage}{\textwidth}
  \protect\hypertarget{module_import}{}{module\_import:}

  \includegraphics[width=4.20833in,height=0.70833in]{diagram/module_import.png}

  \begin{verbatim}
  module_import
         ::= IMPORT IDENTIFIER module_aliasing?
  \end{verbatim}

  Referenciado por:

  \begin{itemize}
      \tightlist%
    \item
      \protect\hyperlink{imports}{imports}
  \end{itemize}

\end{minipage}

\begin{minipage}{\textwidth}
  \protect\hypertarget{feature_import}{}{feature\_import:}

  \includegraphics[width=4.87500in,height=0.83333in]{diagram/feature_import.png}

  \begin{verbatim}
  feature_import
         ::= FROM IDENTIFIER IMPORT feature_name ( ',' feature_name )*
  \end{verbatim}

  Referenciado por:

  \begin{itemize}
      \tightlist%
    \item
      \protect\hyperlink{imports}{imports}
  \end{itemize}

\end{minipage}

\begin{minipage}{\textwidth}
  \protect\hypertarget{feature_name}{}{feature\_name:}

  \includegraphics[width=3.31250in,height=0.70833in]{diagram/feature_name.png}

  \begin{verbatim}
  feature_name
         ::= IDENTIFIER module_aliasing?
  \end{verbatim}

  Referenciado por:

  \begin{itemize}
      \tightlist%
    \item
      \protect\hyperlink{feature_import}{feature\_import}
  \end{itemize}

\end{minipage}

\begin{minipage}{\textwidth}
  \protect\hypertarget{module_aliasing}{}{module\_aliasing:}

  \includegraphics[width=2.08333in,height=0.37500in]{diagram/module_aliasing.png}

  \begin{verbatim}
  module_aliasing
         ::= AS IDENTIFIER
  \end{verbatim}

  Referenciado por:

  \begin{itemize}
      \tightlist%
    \item
      \protect\hyperlink{feature_name}{feature\_name}
    \item
      \protect\hyperlink{module_import}{module\_import}
  \end{itemize}

\end{minipage}

\begin{minipage}{\textwidth}
  \protect\hypertarget{definition}{}{definition:}

  \includegraphics[width=3.66667in,height=0.83333in]{diagram/definition.png}

  \begin{verbatim}
  definition
         ::= EXPORT? ( function_definition | class_definition )
  \end{verbatim}

  Referenciado por:

  \begin{itemize}
      \tightlist%
    \item
      \protect\hyperlink{code}{code}
  \end{itemize}

\end{minipage}

\begin{minipage}{\textwidth}
  \protect\hypertarget{function_definition}{}{function\_definition:}

  \includegraphics[width=5.97917in,height=0.70833in]{diagram/function_definition.png}

  \begin{verbatim}
  function_definition
         ::= LET IDENTIFIER '(' function_params_list? ')' scope
  \end{verbatim}

  Referenciado por:

  \begin{itemize}
      \tightlist%
    \item
      \protect\hyperlink{class_scope}{class\_scope}
    \item
      \protect\hyperlink{definition}{definition}
  \end{itemize}

\end{minipage}

\begin{minipage}{\textwidth}
  \protect\hypertarget{function_params_list}{}{function\_params\_list:}

  \includegraphics[width=4.56250in,height=1.62500in]{diagram/function_params_list.png}

  \begin{verbatim}
  function_params_list
         ::= IDENTIFIER ( ',' IDENTIFIER )* ( ',' variadic_param )?
           | variadic_param
  \end{verbatim}

  Referenciado por:

  \begin{itemize}
      \tightlist%
    \item
      \protect\hyperlink{function_definition}{function\_definition}
  \end{itemize}

\end{minipage}

\begin{minipage}{\textwidth}
  \protect\hypertarget{variadic_param}{}{variadic\_param:}

  \includegraphics[width=2.43750in,height=0.37500in]{diagram/variadic_param.png}

  \begin{verbatim}
  variadic_param
         ::= ELLIPSIS IDENTIFIER
  \end{verbatim}

  Referenciado por:

  \begin{itemize}
      \tightlist%
    \item
      \protect\hyperlink{function_params_list}{function\_params\_list}
  \end{itemize}

\end{minipage}

\begin{minipage}{\textwidth}
  \protect\hypertarget{scope}{}{scope:}

  \includegraphics[width=4.81250in,height=0.54167in]{diagram/scope.png}

  \begin{verbatim}
  scope    ::= LINE_BREAK INDENT command+ DEDENT
  \end{verbatim}

  Referenciado por:

  \begin{itemize}
      \tightlist%
    \item
      \protect\hyperlink{for_statement}{for\_statement}
    \item
      \protect\hyperlink{function_definition}{function\_definition}
    \item
      \protect\hyperlink{if_statement}{if\_statement}
    \item
      \protect\hyperlink{while_statement}{while\_statement}
  \end{itemize}

\end{minipage}

\begin{minipage}{\textwidth}
  \protect\hypertarget{class_definition}{}{class\_definition:}

  \includegraphics[width=3.47917in,height=0.37500in]{diagram/class_definition.png}

  \begin{verbatim}
  class_definition
         ::= CLASS IDENTIFIER class_scope
  \end{verbatim}

  Referenciado por:

  \begin{itemize}
      \tightlist%
    \item
      \protect\hyperlink{definition}{definition}
  \end{itemize}

\end{minipage}

\begin{minipage}{\textwidth}
  \protect\hypertarget{class_scope}{}{class\_scope:}

  \includegraphics[width=5.72917in,height=1.00000in]{diagram/class_scope.png}

  \begin{verbatim}
  class_scope
         ::= LINE_BREAK INDENT ( attr_decl | function_definition )+ DEDENT
  \end{verbatim}

  Referenciado por:

  \begin{itemize}
      \tightlist%
    \item
      \protect\hyperlink{class_definition}{class\_definition}
  \end{itemize}

\end{minipage}

\begin{minipage}{\textwidth}
  \protect\hypertarget{attr_decl}{}{attr\_decl:}

  \includegraphics[width=5.20833in,height=0.70833in]{diagram/attr_decl.png}

  \begin{verbatim}
  attr_decl
         ::= LET IDENTIFIER ( '=' operator )? LINE_BREAK
  \end{verbatim}

  Referenciado por:

  \begin{itemize}
      \tightlist%
    \item
      \protect\hyperlink{class_scope}{class\_scope}
  \end{itemize}

\end{minipage}

\begin{minipage}{\textwidth}
  \protect\hypertarget{command}{}{command:}

  \includegraphics[width=4.27083in,height=2.20833in]{diagram/command.png}

  \begin{verbatim}
  command  ::= ( assignment_expr | function_call | statement | ( THROW | RETURN ) operator ) LINE_BREAK
  \end{verbatim}

  Referenciado por:

  \begin{itemize}
      \tightlist%
    \item
      \protect\hyperlink{code}{code}
    \item
      \protect\hyperlink{scope}{scope}
  \end{itemize}

\end{minipage}

\begin{minipage}{\textwidth}
  \protect\hypertarget{assignment_expr}{}{assignment\_expr:}

  \includegraphics[width=4.02083in,height=0.70833in]{diagram/assignment_expr.png}

  \begin{verbatim}
  assignment_expr
         ::= LET? IDENTIFIER '=' operator
  \end{verbatim}

  Referenciado por:

  \begin{itemize}
      \tightlist%
    \item
      \protect\hyperlink{command}{command}
  \end{itemize}

\end{minipage}

\begin{minipage}{\textwidth}
  \protect\hypertarget{function_call}{}{function\_call:}

  \includegraphics[width=4.81250in,height=1.45833in]{diagram/function_call.png}

  \begin{verbatim}
  function_call
         ::= IDENTIFIER '(' ( ( IDENTIFIER | literal ) ( ',' ( IDENTIFIER | literal ) )* )? ')'
  \end{verbatim}

  Referenciado por:

  \begin{itemize}
      \tightlist%
    \item
      \protect\hyperlink{command}{command}
  \end{itemize}

\end{minipage}

\begin{minipage}{\textwidth}
  \protect\hypertarget{statement}{}{statement:}

  \includegraphics[width=2.25000in,height=1.29167in]{diagram/statement.png}

  \begin{verbatim}
  statement
         ::= if_statement
           | while_statement
           | for_statement
  \end{verbatim}

  Referenciado por:

  \begin{itemize}
      \tightlist%
    \item
      \protect\hyperlink{command}{command}
  \end{itemize}

\end{minipage}

\begin{minipage}{\textwidth}
  \protect\hypertarget{if_statement}{}{if\_statement:}

  \includegraphics[width=4.56250in,height=0.70833in]{diagram/if_statement.png}

  \begin{verbatim}
  if_statement
         ::= IF condition scope ( ELSE scope )?
  \end{verbatim}

  Referenciado por:

  \begin{itemize}
      \tightlist%
    \item
      \protect\hyperlink{statement}{statement}
  \end{itemize}

\end{minipage}

\begin{minipage}{\textwidth}
  \protect\hypertarget{while_statement}{}{while\_statement:}

  \includegraphics[width=2.95833in,height=0.37500in]{diagram/while_statement.png}

  \begin{verbatim}
  while_statement
         ::= WHILE condition scope
  \end{verbatim}

  Referenciado por:

  \begin{itemize}
      \tightlist%
    \item
      \protect\hyperlink{statement}{statement}
  \end{itemize}

\end{minipage}

\begin{minipage}{\textwidth}
  \protect\hypertarget{for_statement}{}{for\_statement:}

  \includegraphics[width=4.43750in,height=0.37500in]{diagram/for_statement.png}

  \begin{verbatim}
  for_statement
         ::= FOR IDENTIFIER IN operator scope
  \end{verbatim}

  Referenciado por:

  \begin{itemize}
      \tightlist%
    \item
      \protect\hyperlink{statement}{statement}
  \end{itemize}

\end{minipage}

\begin{center}\rule{0.5\linewidth}{\linethickness}\end{center}


\clearpage

\section{Arquivos}

Os arquivos FLEX e BISON são respectivamente \texttt{sucuri.l} e
\texttt{sucuri.y}. Exemplos de programas válidos se encontram na pasta
\texttt{examples/}. O código fonte do analisador é \texttt{sucuri.yy.c}.  Os
logs de saída aplicados no exemplo \texttt{examples/geometry.scr} estão no
arquivo \texttt{geometry-parse.ylog}.

\end{document}
